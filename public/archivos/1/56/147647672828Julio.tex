
\documentclass{article}
%%%%%%%%%%%%%%%%%%%%%%%%%%%%%%%%%%%%%%%%%%%%%%%%%%%%%%%%%%%%%%%%%%%%%%%%%%%%%%%%%%%%%%%%%%%%%%%%%%%%%%%%%%%%%%%%%%%%%%%%%%%%%%%%%%%%%%%%%%%%%%%%%%%%%%%%%%%%%%%%%%%%%%%%%%%%%%%%%%%%%%%%%%%%%%%%%%%%%%%%%%%%%%%%%%%%%%%%%%%%%%%%%%%%%%%%%%%%%%%%%%%%%%%%%%%%
%TCIDATA{OutputFilter=LATEX.DLL}
%TCIDATA{Version=5.50.0.2953}
%TCIDATA{<META NAME="SaveForMode" CONTENT="1">}
%TCIDATA{BibliographyScheme=Manual}
%TCIDATA{Created=Thursday, July 28, 2016 17:19:23}
%TCIDATA{LastRevised=Thursday, July 28, 2016 19:12:26}
%TCIDATA{<META NAME="GraphicsSave" CONTENT="32">}
%TCIDATA{<META NAME="DocumentShell" CONTENT="Standard LaTeX\Blank - Standard LaTeX Article">}
%TCIDATA{CSTFile=40 LaTeX article.cst}

\newtheorem{theorem}{Theorem}
\newtheorem{acknowledgement}[theorem]{Acknowledgement}
\newtheorem{algorithm}[theorem]{Algorithm}
\newtheorem{axiom}[theorem]{Axiom}
\newtheorem{case}[theorem]{Case}
\newtheorem{claim}[theorem]{Claim}
\newtheorem{conclusion}[theorem]{Conclusion}
\newtheorem{condition}[theorem]{Condition}
\newtheorem{conjecture}[theorem]{Conjecture}
\newtheorem{corollary}[theorem]{Corollary}
\newtheorem{criterion}[theorem]{Criterion}
\newtheorem{definition}[theorem]{Definition}
\newtheorem{example}[theorem]{Example}
\newtheorem{exercise}[theorem]{Exercise}
\newtheorem{lemma}[theorem]{Lemma}
\newtheorem{notation}[theorem]{Notation}
\newtheorem{problem}[theorem]{Problem}
\newtheorem{proposition}[theorem]{Proposition}
\newtheorem{remark}[theorem]{Remark}
\newtheorem{solution}[theorem]{Solution}
\newtheorem{summary}[theorem]{Summary}
\newenvironment{proof}[1][Proof]{\noindent\textbf{#1.} }{\ \rule{0.5em}{0.5em}}
\input{tcilatex}

\begin{document}


Teorema de Green

$\doint\limits_{C}Pdx+Qdy=\int \int_{D}\left( \frac{\partial P}{\partial y}-%
\frac{\partial Q}{\partial x}\right) dA$

donde $C$ corre en sentido horario

-------------------------------------------------------

Evaluar $\doint\limits_{C}x^{2}dx-xydy$ donde $C$ es el tri\'{a}ngulo que
forman los puntos $\left( 0,0\right) ,\left( 1,0\right) $ y $\left(
1,1\right) $.

\FRAME{dtbpFX}{2.2946in}{1.5303in}{0pt}{}{}{Plot}{\special{language
"Scientific Word";type "MAPLEPLOT";width 2.2946in;height 1.5303in;depth
0pt;display "USEDEF";plot_snapshots TRUE;mustRecompute FALSE;lastEngine
"MuPAD";xmin "-5";xmax "5";xviewmin "-0.0001";xviewmax "1.0001";yviewmin
"-0.0001";yviewmax "1.0001";plottype 4;axesFont "Times New
Roman,12,0000000000,useDefault,normal";numpoints 100;plotstyle
"patchnogrid";axesstyle "normal";axestips FALSE;xis \TEXUX{x};var1name
\TEXUX{$x$};function \TEXUX{$\left( 0,0\right) $};linecolor "blue";linestyle
1;pointplot TRUE;pointstyle "box";linethickness 1;lineAttributes
"Solid";curveColor "[flat::RGB:0x000000ff]";curveStyle "Point";function
\TEXUX{$\left( 1,0\right) $};linecolor "blue";linestyle 1;pointplot
TRUE;pointstyle "box";linethickness 1;lineAttributes "Solid";curveColor
"[flat::RGB:0x000000ff]";curveStyle "Point";function \TEXUX{$\left(
1,1\right) $};linecolor "blue";linestyle 1;pointplot TRUE;pointstyle
"box";linethickness 1;lineAttributes "Solid";curveColor
"[flat::RGB:0x000000ff]";curveStyle "Point";VCamFile
'OB1QSM06.xvz';valid_file "T";tempfilename
'OB1QSM02.wmf';tempfile-properties "XPR";}}

Primer tramo:

$\int_{0}^{1}x^{2}dx-x\left( 0\right) \left( 0\right)
=\int_{0}^{1}x^{2}dx=\allowbreak \frac{1}{3}$

Segundo tramo:

$\int_{0}^{1}\left( 1\right) ^{2}\left( 0\right)
-1ydy=\int_{0}^{1}-ydy=\allowbreak -\frac{1}{2}$

Tercer tramo:

$\int_{1}^{0}x^{2}dx-xxdx=0$

El resultado final es la suma de los 3:

$\frac{1}{3}-\frac{1}{2}+0=\allowbreak -\frac{1}{6}$

-------------------------------

$\doint\limits_{C}Pdx+Qdy=\int \int_{D}\left( \frac{\partial P}{\partial y}-%
\frac{\partial Q}{\partial x}\right) dA$

$\doint\limits_{C}x^{2}dx-xydy=\int \int_{D}\left( 0-\left( -y\right)
\right) dA=\int \int_{D}ydA$

$\int_{0}^{1}\int_{0}^{x}ydydx=\allowbreak \frac{1}{6}$

Como $C$ corre antihorario el verdadero resultado es $-\frac{1}{6}$

$F=\left[ 
\begin{array}{c}
x^{2} \\ 
-xy%
\end{array}%
\right] $\FRAME{dtbpFX}{3.6898in}{2.4602in}{0pt}{}{}{Plot}{\special{language
"Scientific Word";type "MAPLEPLOT";width 3.6898in;height 2.4602in;depth
0pt;display "USEDEF";plot_snapshots TRUE;mustRecompute FALSE;lastEngine
"MuPAD";xmin "0";xmax "1";ymin "0";ymax "1";xviewmin "-0.0001";xviewmax
"1.0001";yviewmin "-0.0001";yviewmax "1.0001";plottype 11;axesFont "Times
New Roman,12,0000000000,useDefault,normal";num-x-gridlines
10;num-y-gridlines 10;plotstyle "patchnogrid";axesstyle "normal";axestips
FALSE;xis \TEXUX{x};yis \TEXUX{y};var1name \TEXUX{$x$};var2name
\TEXUX{$y$};function \TEXUX{$\left[
\MATRIX{1,2}{c}\VR{,,c,,,}{,,,,,}\HR{,,}\CELL{x^{2}}\CELL{-xy}\right]
$};linecolor "black";linestyle 1;pointplot TRUE;pointstyle
"point";linethickness 1;lineAttributes "Solid";var1range "0,1";var2range
"0,1";num-x-gridlines 10;num-y-gridlines 10;curveColor
"[flat::RGB:0000000000]";curveStyle "Point";rangeset"XY";VCamFile
'OB1R6309.xvz';valid_file "T";tempfilename
'OB1R5U03.wmf';tempfile-properties "XPR";}}

-----------------------------------------------------------------------------------

Eval\'{u}e $\doint\limits_{C}x^{2}ydx+3xdy$ donde $C$ es el c\'{\i}rculo en
el origen de radio 2 en sentido horario.\qquad $x^{2}+y^{2}=r^{2}$

$x^{2}+y^{2}=4$

$y=\pm \sqrt{4-x^{2}}$

$dy=\pm \frac{1}{2}\left( 4-x^{2}\right) ^{-\frac{1}{2}}\left( -2x\right)
=\pm -\frac{x}{\sqrt{4-x^{2}}}dx$

$\int_{-2}^{2}\left( x^{2}\sqrt{4-x^{2}}+3x\left( -\frac{x}{\sqrt{4-x^{2}}}%
\right) \right) dx=\allowbreak -4\pi $

$\int_{2}^{-2}\left( -x^{2}\sqrt{4-x^{2}}+3x\left( \frac{x}{\sqrt{4-x^{2}}}%
\right) \right) dx=\allowbreak -4\pi $

$-4\pi -4\pi =\allowbreak -8\pi $

Ahora con Green en $\left( x,y\right) $

$\doint\limits_{C}x^{2}ydx+3xdy=\int \int_{D}\left( x^{2}-3\right) dA$

$\int_{-2}^{2}\int_{-\sqrt{4-x^{2}}}^{\sqrt{4-x^{2}}}\left( x^{2}-3\right)
dydx=\allowbreak -8\pi $

Ahora haremos ambas (l\'{\i}nea y \'{a}rea) en coordenadas polares:

$x=2\cos \theta \qquad \qquad y=2\sin \theta $

$dx=-2\sin \theta d\theta \qquad dy=2\cos \theta d\theta $

$\doint\limits_{C}x^{2}ydx+3xdy=\int_{2\pi }^{0}\left( 2\cos \theta \right)
^{2}\left( 2\sin \theta \right) \left( -2\sin \theta d\theta \right)
+3\left( 2\cos \theta \right) \left( 2\cos \theta d\theta \right) $

$=\int_{2\pi }^{0}\left( \left( 2\cos \theta \right) ^{2}\left( 2\sin \theta
\right) \left( -2\sin \theta \right) +3\left( 2\cos \theta \right) \left(
2\cos \theta \right) \right) d\theta =\allowbreak -8\pi $

Ahora Green en polares:

$x=r\cos \theta \qquad \qquad y=r\sin \theta $

$dA=rdrd\theta $

$\int \int_{D}\left( x^{2}-3\right) dA=\int_{0}^{2\pi }\int_{0}^{2}\left(
\left( r\cos \theta \right) ^{2}-3\right) rdrd\theta =\allowbreak -8\pi $

------------------------------------------------------------------------

Operaciones sobre funciones vectoriales y escalares

- Gradiente\qquad $\nabla f=F\qquad \qquad $\qquad Esc-\TEXTsymbol{>}Vect

\qquad \qquad \qquad \qquad $\func{grad}f$

- Divergencia\qquad $\nabla \cdot F=f$\qquad \qquad \qquad Vect-\TEXTsymbol{>%
}Esc

\qquad \qquad \qquad \qquad \qquad $\func{div}F$

- Rotacional\qquad $\nabla \times F=F$\qquad \qquad \qquad Vect-\TEXTsymbol{>%
}Vect

\qquad \qquad \qquad \qquad \qquad $\func{curl}F\qquad \qquad $rot$F$

- Laplaciano\qquad $\nabla ^{2}f=f$\qquad \qquad \qquad \qquad Esc-%
\TEXTsymbol{>}Esc

\qquad \qquad \qquad \qquad \qquad lapl$f$

-----------------------------------------------------------------

$\nabla =\left[ 
\begin{array}{c}
\frac{\partial }{\partial x} \\ 
\frac{\partial }{\partial y} \\ 
\frac{\partial }{\partial z}%
\end{array}%
\right] $

$\nabla f=\left[ 
\begin{array}{c}
\frac{\partial }{\partial x} \\ 
\frac{\partial }{\partial y} \\ 
\frac{\partial }{\partial z}%
\end{array}%
\right] f=\left[ 
\begin{array}{c}
\frac{\partial }{\partial x}f \\ 
\frac{\partial }{\partial y}f \\ 
\frac{\partial }{\partial z}f%
\end{array}%
\right] $

$\nabla \cdot F=\left[ 
\begin{array}{c}
\frac{\partial }{\partial x} \\ 
\frac{\partial }{\partial y} \\ 
\frac{\partial }{\partial z}%
\end{array}%
\right] \cdot \left[ 
\begin{array}{c}
F_{1} \\ 
F_{2} \\ 
F_{3}%
\end{array}%
\right] =\frac{\partial }{\partial x}F_{1}+\frac{\partial }{\partial y}F_{2}+%
\frac{\partial }{\partial z}F_{3}$

$\nabla \times F=\left[ 
\begin{array}{c}
\frac{\partial }{\partial x} \\ 
\frac{\partial }{\partial y} \\ 
\frac{\partial }{\partial z}%
\end{array}%
\right] \times \left[ 
\begin{array}{c}
F_{1} \\ 
F_{2} \\ 
F_{3}%
\end{array}%
\right] =\left[ 
\begin{array}{c}
\frac{\partial }{\partial y}F_{3}-\frac{\partial }{\partial z}F_{2} \\ 
\frac{\partial }{\partial z}F_{1}-\frac{\partial }{\partial x}F_{3} \\ 
\frac{\partial }{\partial x}F_{2}-\frac{\partial }{\partial y}F_{1}%
\end{array}%
\right] $

$\nabla ^{2}f=\nabla \cdot \left( \nabla f\right) =\left[ 
\begin{array}{c}
\frac{\partial }{\partial x} \\ 
\frac{\partial }{\partial y} \\ 
\frac{\partial }{\partial z}%
\end{array}%
\right] \cdot \left[ 
\begin{array}{c}
\frac{\partial }{\partial x} \\ 
\frac{\partial }{\partial y} \\ 
\frac{\partial }{\partial z}%
\end{array}%
\right] f=\left[ 
\begin{array}{c}
\frac{\partial }{\partial x} \\ 
\frac{\partial }{\partial y} \\ 
\frac{\partial }{\partial z}%
\end{array}%
\right] \cdot \left[ 
\begin{array}{c}
\frac{\partial }{\partial x}f \\ 
\frac{\partial }{\partial y}f \\ 
\frac{\partial }{\partial z}f%
\end{array}%
\right] =\frac{\partial ^{2}}{\partial x^{2}}f+\frac{\partial ^{2}}{\partial
y^{2}}f+\frac{\partial ^{2}}{\partial z^{2}}f$

---------------------------------------------------------------

$\func{grad}\left( \func{div}F\right) \qquad \qquad $Si, vector

$\func{div}\left( \func{div}F\right) \qquad \qquad $No, la segunda
divergencia no es posible

$\nabla \left( \nabla \cdot \left( \nabla \times F\right) \right) =\func{grad%
}\left( \func{div}\left( \func{curl}F\right) \right) \qquad \qquad $Si,
vector

$\nabla \cdot \left( \nabla \left( \nabla \times F\right) \right) \qquad $%
No, porque no se puede calcular un gradiente de un vector

-----------------------------------------------------------------------------------

$F=x^{2}z$i$+\sin y$j$-3xz$k

$F=\left[ 
\begin{array}{c}
x^{2}z \\ 
\sin y \\ 
-3xz%
\end{array}%
\right] $

Calcular

$\func{curl}\left[ 
\begin{array}{c}
x^{2}z \\ 
\sin y \\ 
-3xz%
\end{array}%
\right] =\allowbreak \left[ 
\begin{array}{c}
0 \\ 
x^{2}+3z \\ 
0%
\end{array}%
\right] $

$\func{div}\left[ 
\begin{array}{c}
x^{2}z \\ 
\sin y \\ 
-3xz%
\end{array}%
\right] =\allowbreak \cos y-3x+2xz$

$\nabla \left( \nabla \cdot \left[ 
\begin{array}{c}
x^{2}z \\ 
\sin y \\ 
-3xz%
\end{array}%
\right] \right) =\allowbreak \left[ 
\begin{array}{c}
2z-3 \\ 
-\sin y \\ 
2x%
\end{array}%
\right] $

\end{document}
