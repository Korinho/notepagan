
\documentclass{article}
%%%%%%%%%%%%%%%%%%%%%%%%%%%%%%%%%%%%%%%%%%%%%%%%%%%%%%%%%%%%%%%%%%%%%%%%%%%%%%%%%%%%%%%%%%%%%%%%%%%%%%%%%%%%%%%%%%%%%%%%%%%%%%%%%%%%%%%%%%%%%%%%%%%%%%%%%%%%%%%%%%%%%%%%%%%%%%%%%%%%%%%%%%%%%%%%%%%%%%%%%%%%%%%%%%%%%%%%%%%%%%%%%%%%%%%%%%%%%%%%%%%%%%%%%%%%
%TCIDATA{OutputFilter=LATEX.DLL}
%TCIDATA{Version=5.50.0.2953}
%TCIDATA{<META NAME="SaveForMode" CONTENT="1">}
%TCIDATA{BibliographyScheme=Manual}
%TCIDATA{Created=Tuesday, July 26, 2016 17:10:47}
%TCIDATA{LastRevised=Tuesday, July 26, 2016 19:18:25}
%TCIDATA{<META NAME="GraphicsSave" CONTENT="32">}
%TCIDATA{<META NAME="DocumentShell" CONTENT="Standard LaTeX\Blank - Standard LaTeX Article">}
%TCIDATA{CSTFile=40 LaTeX article.cst}

\newtheorem{theorem}{Theorem}
\newtheorem{acknowledgement}[theorem]{Acknowledgement}
\newtheorem{algorithm}[theorem]{Algorithm}
\newtheorem{axiom}[theorem]{Axiom}
\newtheorem{case}[theorem]{Case}
\newtheorem{claim}[theorem]{Claim}
\newtheorem{conclusion}[theorem]{Conclusion}
\newtheorem{condition}[theorem]{Condition}
\newtheorem{conjecture}[theorem]{Conjecture}
\newtheorem{corollary}[theorem]{Corollary}
\newtheorem{criterion}[theorem]{Criterion}
\newtheorem{definition}[theorem]{Definition}
\newtheorem{example}[theorem]{Example}
\newtheorem{exercise}[theorem]{Exercise}
\newtheorem{lemma}[theorem]{Lemma}
\newtheorem{notation}[theorem]{Notation}
\newtheorem{problem}[theorem]{Problem}
\newtheorem{proposition}[theorem]{Proposition}
\newtheorem{remark}[theorem]{Remark}
\newtheorem{solution}[theorem]{Solution}
\newtheorem{summary}[theorem]{Summary}
\newenvironment{proof}[1][Proof]{\noindent\textbf{#1.} }{\ \rule{0.5em}{0.5em}}
\input{tcilatex}

\begin{document}


INTEGRALES\ DE\ L\'{I}NEA

$u\cdot v=\left\Vert u\right\Vert \left\Vert v\right\Vert \cos \theta $

$dr=\left[ 
\begin{array}{c}
dx \\ 
dy%
\end{array}%
\right] $

$\int_{C}F\cdot dr,$ donde $F\left( x,y\right) =\left[ 
\begin{array}{c}
x-y \\ 
x^{2}%
\end{array}%
\right] $ y $C$ es la recta que va de $\left( 0,0\right) $ a $\left(
2,2\right) $

$y=x\qquad dy=dx$

\bigskip $\int_{C}\left( \left( x-y\right) \text{i}+x^{2}\text{j}\right)
\cdot \left( dx\text{i}+dy\text{j}\right) $

\bigskip $\int_{C}\left[ 
\begin{array}{c}
x-y \\ 
x^{2}%
\end{array}%
\right] \cdot \left[ 
\begin{array}{c}
dx \\ 
dy%
\end{array}%
\right] =\int \left( x-y\right) dx+\int x^{2}dy$

$=\int \left( x-x\right) dx+\int x^{2}dx=\int_{0}^{2}x^{2}dx=\allowbreak 
\frac{8}{3}$

-----------------------------------------------

$\int_{C}F\cdot dr$ donde $F=\left[ 
\begin{array}{c}
x^{2}+y \\ 
x-y%
\end{array}%
\right] $ y $C$ es la curva $y=x^{2}$ desde $\left( 0,0\right) $ hasta $%
\left( 1,1\right) $

$y=x^{2}\qquad dy=2xdx$

$\int_{C}\left( x^{2}+y\right) dx+\left( x-y\right) dy$

$\int_{C}\left( x^{2}+x^{2}\right) dx+\left( x-x^{2}\right) 2xdx$

$\int_{0}^{1}\left( \left( x^{2}+x^{2}\right) +\left( x-x^{2}\right)
2x\right) dx=\allowbreak \frac{5}{6}$

-----------------------------------------------

$r=$\bigskip $\left[ 
\begin{array}{c}
x \\ 
y \\ 
z%
\end{array}%
\right] $

------------------------------------------------

$\int_{C}F\cdot dr$ donde $F=\left[ 
\begin{array}{c}
x-z \\ 
x-y \\ 
2y+1%
\end{array}%
\right] $ y $C$ es la recta que va de $\left( 0,2,-1\right) $ a $\left(
3,2,6\right) $

\FRAME{dtbpFX}{2.271in}{1.5143in}{0pt}{}{}{Plot}{\special{language
"Scientific Word";type "MAPLEPLOT";width 2.271in;height 1.5143in;depth
0pt;display "USEDEF";plot_snapshots TRUE;mustRecompute TRUE;lastEngine
"MuPAD";xmin "0";xmax "1";ymin "-5";ymax "5";xviewmin "-3E-10";xviewmax
"3.0000000003";yviewmin "1";yviewmax "3";zviewmin "-1.0000000007";zviewmax
"6.0000000007";phi 45;theta 45;plottype 5;axesFont "Times New
Roman,12,0000000000,useDefault,normal";num-x-gridlines 25;num-y-gridlines
25;plotstyle "patch";axesstyle "normal";axestips FALSE;plotshading
"XYZ";lighting 0;xis \TEXUX{x};yis \TEXUX{y};var1name \TEXUX{$x$};var2name
\TEXUX{$y$};function \TEXUX{$\left[
\MATRIX{1,3}{c}\VR{,,c,,,}{,,,,,}\HR{,,,}\CELL{3t}\CELL{2}\CELL{7t-1}\right]
$};linestyle 1;pointstyle "point";linethickness 1;lineAttributes
"Solid";curveStyle "Line";var1range "0,1";var2range "-5,5";surfaceColor
"[linear:XYZ:RGB:0x00ff0000:0x000000ff]";num-x-gridlines 25;num-y-gridlines
25;rangeset"X";function \TEXUX{$\left( 0,2,-1\right) $};linestyle
1;pointplot TRUE;pointstyle "box";linethickness 1;lineAttributes
"Solid";curveStyle "Point";surfaceColor
"[linear:XYZ:RGB:0x0000ff00:0x00008000]";function \TEXUX{$\left(
3,2,6\right) $};linestyle 1;pointplot TRUE;pointstyle "box";linethickness
1;lineAttributes "Solid";curveStyle "Point";surfaceColor
"[linear:XYZ:RGB:0x00008000:0x0000ff00]";VCamFile 'OAY2VK0D.xvz';valid_file
"T";tempfilename 'OAY2VJ05.wmf';tempfile-properties "XPR";}}

$x=3t+0=\allowbreak 3t\qquad \qquad \qquad dx=3dt$

$y=0t+2=\allowbreak 2\qquad \qquad \qquad dy=0$

$z=7t-1\qquad \qquad \qquad \qquad dz=7dt\qquad \qquad 0\leq t\leq 1\left[ 
\begin{array}{c}
3t \\ 
2 \\ 
7t-1%
\end{array}%
\right] $

$\int \left( x-z\right) dx+\left( x-y\right) dy+\left( 2y+1\right) dz$

$\int_{0}^{1}\left( \allowbreak 3t-\left( 7t-1\right) \right) \left(
3dt\right) +\left( \allowbreak 3t-2\right) \left( 0\right) +\left( 2\left(
2\right) +1\right) \left( 7dt\right) $

$\int_{0}^{1}\left( \left( \allowbreak 3t-\left( 7t-1\right) \right) \left(
3\right) +\left( 2\left( 2\right) +1\right) \left( 7\right) \right)
dt=\allowbreak 32$

$\left( a,b,c\right) ->\left( d,e,f\right) $

$x=\left( d-a\right) t+a\qquad y=\left( e-b\right) t+b\qquad \qquad z=\left(
f-c\right) t+c$

-----------------------------------------------------------------------------

Integral de l\'{\i}nea sobre funciones escalares

$ds=\left\Vert dr\right\Vert $

$ds=\sqrt{\left( dx\right) ^{2}+\left( dy\right) ^{2}}$

$ds=\sqrt{\left( dx\right) ^{2}+\left( dx\frac{dy}{dx}\right) ^{2}}=\sqrt{%
\left( dx\right) ^{2}+\left( dx\right) ^{2}\left( \frac{dy}{dx}\right) ^{2}}=
$

$=\sqrt{\left( 1+\left( \frac{dy}{dx}\right) ^{2}\right) \left( dx\right)
^{2}}=\sqrt{1+\left( \frac{dy}{dx}\right) ^{2}}dx$

$ds=\sqrt{1+\left( \frac{dx}{dy}\right) ^{2}}dy$

$ds=\sqrt{\left( \frac{dx}{dt}dt\right) ^{2}+\left( \frac{dy}{dt}dt\right)
^{2}}=\sqrt{\left( \frac{dx}{dt}\right) ^{2}\left( dt\right) ^{2}+\left( 
\frac{dy}{dt}\right) ^{2}\left( dt\right) ^{2}}$

$=\sqrt{\left( \left( \frac{dx}{dt}\right) ^{2}+\left( \frac{dy}{dt}\right)
^{2}\right) \left( dt\right) ^{2}}=\sqrt{\left( \frac{dx}{dt}\right)
^{2}+\left( \frac{dy}{dt}\right) ^{2}}dt$

$\int_{C}yds$, donde $C$ es $y=\sin x$ desde $\left( 0,0\right) $ a $\left(
\pi ,0\right) $

$\frac{dy}{dx}=\cos x$

$\int_{C}y\sqrt{1+\left( \frac{dy}{dx}\right) ^{2}}dx=\int_{0}^{\pi }\sin x%
\sqrt{1+\left( \cos x\right) ^{2}}dx=\allowbreak 1.\,\allowbreak
338\,487\,7\allowbreak $

------------------------------------------------

$\int_{C}\left( x+y\right) ds$, donde $C$ es la curva $r=\left( t,t^{2},%
\sqrt{t}\right) $ con $1\leq t\leq 4$

$x=t\qquad \qquad \frac{dx}{dt}=1\qquad \qquad $

$y=t^{2}\qquad \qquad \frac{dy}{dt}=2t$

$z=\sqrt{t}\qquad \qquad \frac{dz}{dt}=\frac{1}{2\sqrt{t}}$

$\int_{C}\left( x+y\right) \sqrt{\left( \frac{dx}{dt}\right) ^{2}+\left( 
\frac{dy}{dt}\right) ^{2}+\left( \frac{dz}{dt}\right) ^{2}}dt$

$\int_{1}^{4}\left( t+t^{2}\right) \sqrt{\left( 1\right) ^{2}+\left(
2t\right) ^{2}+\left( \frac{1}{2\sqrt{t}}\right) ^{2}}dt=\allowbreak
172.\,\allowbreak 357\,79\allowbreak $

------------------------------------------------

\textquestiondown C\'{o}mo hacer una integral de l\'{\i}nea cuando el campo
es conservativo?

$\int_{C}F\cdot dr$, donde $F=\left[ 
\begin{array}{c}
2xy^{2} \\ 
2x^{2}y%
\end{array}%
\right] $, donde $C$ es $y=2x^{2}-1$ con $0\leq x\leq 1$

$\left( 0,-1\right) $ a $\left( 1,1\right) $

M\'{e}todo convencional:

$y=2x^{2}-1,dy=4xdx$

$\int 2xy^{2}dx+2x^{2}ydy=\int_{0}^{1}2x\left( 2x^{2}-1\right)
^{2}dx+2x^{2}\left( 2x^{2}-1\right) 4xdx=$

$=\int_{0}^{1}\left( 2x\left( 2x^{2}-1\right) ^{2}+2x^{2}\left(
2x^{2}-1\right) 4x\right) dx=\allowbreak 1$

Verifiquemos si F es conservativo:

$\frac{\partial }{\partial y}2xy^{2}=\allowbreak 4xy\qquad \qquad \qquad
\qquad \qquad \frac{\partial }{\partial x}2x^{2}y=\allowbreak 4xy\qquad SI\
ES$

$\int 2xy^{2}dx=\allowbreak x^{2}y^{2}\qquad \qquad \qquad \qquad \qquad
\int 2x^{2}ydy=\allowbreak x^{2}y^{2}$

$f\left( x,y\right) =x^{2}y^{2}$

$\int_{C}F\cdot dr=\left[ x^{2}y^{2}\right] _{\left( 1,1\right) }-\left[
x^{2}y^{2}\right] _{\left( 0,-1\right) }=\left( 1^{2}1^{2}\right) -\left(
0^{2}\left( -1\right) ^{2}\right) =\allowbreak 1$

\end{document}
